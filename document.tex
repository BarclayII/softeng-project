\documentclass[CJK,utf8]{ctexrep}

% Title Page
\title{
	考试招生录取系统\\
	\large 需求分析与结构化设计文档
}
\author{
	\begin{tabular}{ll}
		12307130129 & 刘宇达 \\
		11307120059 & 王家威 \\
		12307130222 & 李砚业 \\
		11307110008 & 甘全 \\
	\end{tabular}
}


\begin{document}
\maketitle

\section*{引言}

\subsection*{系统概述}
考试招生录取系统通过信息化的方式涵盖了招生录取管理系统填报志愿、招生、投档等
环节,提供了如下功能:
\begin{enumerate}
	\item 院校招生计划管理及发布
	\item 成绩管理及发布
	\item 志愿填报管理
	\item 调档分数线的划定
	\item 招生过程的维护,包括统招、调招、特招、退档、补录等
	\item 录取名册及各个考生录取信息的发布
\end{enumerate}

%直接抄了。。
系统流程以数据流图的方式展现,并辅以数据字典对数据流图中的所有名词加以解释。
与此同时,数据字典中规定了所有数据流的组成项,每个数据项规定了相应的取值类型和范围,
使系统不易出错、可靠性高;而加工图进一步描述了数据流的变化过程,简明易懂。
最后,通过结构图展现了系统由哪些模块组成,以及模块之间的调用关系,
使整套系统的构建更为清晰。

\subsection*{文档概述}
%继续抄。。
本文档详细描述了考试招生录取系统的需求规约,为进一步设计及具体实现奠定了基础。
%参考文献就不要了吧

\section*{项目概述}

\subsection*{目标}
%其实就是变换了一下需求的用语。。
将信息化应用于考试招生录取系统中,方便统一协调考生、招生办、院校之间的关系,提高
办公效率。

\subsection*{用户特点}
本系统由于面向考生、招生办及院校,对考生将来人生规划及发展,以及院校生源的质量把关
有重要影响,
%继续堂而皇之地抄。。
因此要求可靠性较高。对于各类关键数据,设置权限,保证不会被恶意修改。
此外,各项功能差别较大,各模块所需要的操作各不相同,需要分别进行针对性的开发,
才能更好地利用资源。

\end{document}          
